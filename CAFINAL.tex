\documentclass{article}
\usepackage{graphicx, amssymb}
\usepackage{amsmath}
\usepackage{amsfonts}
\usepackage{amsthm}
\usepackage{kotex}
\usepackage{bm}
\usepackage{hyperref}
\usepackage{xcolor}
\usepackage{mathrsfs}
\usepackage{mathtools}
\usepackage{physics}
\usepackage{tikz}
\usetikzlibrary{decorations.markings}

\textwidth 6.5 truein 
\oddsidemargin 0 truein 
\evensidemargin -0.50 truein 
\topmargin -.5 truein 
\textheight 8.5in

\DeclareMathOperator{\cc}{\mathbb{C}}
\DeclareMathOperator{\rr}{\mathbb{R}}
\DeclareMathOperator{\bA}{\mathbb{A}}
\DeclareMathOperator{\fra}{\mathfrak{a}}
\DeclareMathOperator{\frb}{\mathfrak{b}}
\DeclareMathOperator{\frm}{\mathfrak{m}}
\DeclareMathOperator{\frp}{\mathfrak{p}}
\DeclareMathOperator{\slin}{\mathfrak{sl}}
\DeclareMathOperator{\Lie}{\mathsf{Lie}}
\DeclareMathOperator{\Alg}{\mathsf{Alg}}
\DeclareMathOperator{\Spec}{\mathrm{Spec}}
\DeclareMathOperator{\End}{\mathrm{End}}
\DeclareMathOperator{\rad}{\mathrm{rad}}
\newcommand*\Laplace{\mathop{}\!\mathbin\bigtriangleup}
\newcommand{\id}{\mathrm{id}}
\newcommand{\Hom}{\mathrm{Hom}}
\newcommand{\Sch}{\mathbf{Sch}}
\newcommand{\Ring}{\mathbf{Ring}}
\newcommand{\T}{\mathcal{T}}
\newcommand{\B}{\mathcal{B}}
\newcommand{\Mod}[1]{\ (\mathrm{mod}\ #1)}
\newtheorem{lemma}{Lemma}
\newtheorem{theorem}{Theorem}
\newtheorem{proposition}{Proposition}

\begin{document}


\title{Complex Analysis - Final}
\author{SungBin Park, 20150462}

\maketitle
\section{Introduction}
이 수업을 듣기 2년 반 전, 학부 응용복소함수론을 교수님께 처음으로 배울 때의 느낌을 아직까지도 잊히지 않습니다. 당시 체감 수업 진도도 빨랐었고, 해석학의 엄밀함에 대해 익숙하지 않았기 때문에 수업 초반부터 수업 진도를 따라가기 쉬웠던 것은 아니지만, 같은 이유로 응용복소함수론을 통해 수학의 엄밀함에 대해 직접 체험하면서 수학을 하는 재미를 배울 수 있었습니다. 그 수업을 들은 이후로 2년 동안 학부 해석과 대수, 위상 과목을 듣고, 그 후 다시 복소함수론을 공부하니 예전엔 이해하지 못했던 해석학적 접근 방식과 논리들이 눈에 보이기 시작하여 한 한기 동안 수강하면서 복소함수의 흥미로운 부분들을 볼 수 있었습니다. 이번 학기에 배운 것을 정리하면서 인상 깊었던 것들을 정리하는 차원에서 글을 써보도록 하겠습니다.
\section{Introduction to Complex Analysis: Cauchy-Goursat Theorem, Cauchy Riemann equation}
Cauchy-Goursat Theorem과 Cauchy Integral Formula의 statement는 다음과 같습니다.(Complex Variables and Applications, James Ward Brown/Ruel V. Churchill에서 가져왔습니다.)
\begin{theorem}
(Cauchy-Goursat Theorem) If a function $f$ is analytic at all points interior and on a simple closed contour $C$, then
\begin{equation*}
\int_C f(z) dz=0.
\end{equation*}
\end{theorem}
\begin{theorem}
(Cauchy Integral Formula) Let $f$ be analytic everywhere inside and on a simple closed coutour $C$, taken in the positive sense. If $z_0$ is any point interior to $C$, then
\begin{equation*}
f(z_0)=\frac{1}{2\pi i }\int_C \frac{f(z)dz}{z-z_0}.
\end{equation*}
\end{theorem}
이 정리들은 학부에서도 중요하게 다루어지고, 증명도 해석학에 익숙한 사람이면 쉽게 따라갈 수 있는 정리들입니다. 하지만 제가 이 글에서 이 두 정리를 먼저 적은 이유는 이 두 정리가 복소해석의 시작을 알림과 동시에 중심축을 담당하는 정리라고 생각하기 때문입니다. 실변수함수론에서 측도 가능 함수(Measurable function)을 처음 배울 때 그 함수의 특이점(Singularity) 또는 영점(zero)이 어떻게 되는지는 그 점들이 르벡 적분에 영향을 주지 않는 이상 딱히 신경쓰지 않습니다. 또한 더 나아가 측도 가능 함수에 미분 값을 부여하기 위해 약미분(Weak derivative)를 도입하여 함수가 어떻게 움직이는지에 관심을 가집니다. 그와 다르게 복소함수론에서 다루는 함수는 무한번 미분 가능할 뿐만 아니라 함수의 급수로 표현 가능한 함수들을 다루고 그 함수들의 특이점과 영점을 집중하여 보는데, 이것을 가능하게 하는 정리가 바로 위의 두 정리라고 생각합니다. 즉 위의 두 정리는 함수의 성질들을 Contour 적분으로 알아낼 있게 하는 정리입니다. 더 나아가 Cauchy Integral Formula로 부터 파생되는 정리와 그 정리들이 담고 있는 의미 또한 많습니다. 간단하게는 유수(Residue) 정리를 통해 이상 적분의 값을 계산할 수 있고, 특이점들을 Removable pole, order m pole, Essential pole로 분류하여 함수의 특성을 분석할 수 있습니다. 또한 Infinite에서 특이점 분류를 통해 함수의 형태까지도 결정할 수 있으며, 특히 meromorphic function $f$가 meromorphic at $\infty$ 또한 만족한다면 분수와 분모가 다항식 꼴인 함수의 형태를 가진다는 것은 polynomial 함수에 집중하는 대수기하와의 연관성을 암시하는 부분이라고 생각합니다. 더 나아가 위상 적으로는 Multiply connected set을 분석하는 기초가 됩니다. 이와 같은 중요성을 가지고 있기 때문에 저에게 복소함수론에서 가장 중요한 정리를 뽑으라고 하면 저는 Cauchy-Goursat Theorem과 더불어 Cauchy-Integral Formula를 선택할 것입니다.
\section{Weierstrass Factorization Theorem, Mittag-Leffler Theorem, Riemann Zata function}
Weierstrass Factorization Theorem의 statement는 다음과 같습니다.(Function Theory of One Complex Variable Robert E. Greene, Steven G. Krantz에서 가져왔습니다.)
\begin{theorem}
(Weierstrass Factorization Theorem) Let $f$ be an entire function. Suppose that $f$ vanishes to order $m$ at $0$, $m\geq 0$. Let $\{a_n\}$ be the other zeros of $f$, listed with multiplicities. Then there is an entire function $g$ such that
\begin{equation*}
f(z)=z^m e^{g(z)}\prod\limits_{n=1}^\infty E_{n-1}\left(\frac{z}{a_n}\right)
\end{equation*}
where
\begin{equation*}
E_n(z)=(1-z)\exp\left(z+\frac{z^2}{2}+\cdots+\frac{z^n}{n}\right).
\end{equation*}
\end{theorem}
Weierstrass Factorization Theorem은 Holomorphic fucnction의 영점들을 통해 함수를 표현할 수 있게 하며, 무한 곱의 형태를 띄고 있기 때문에 무한 곱에 대한 엄밀한 정의가 선행되어야 하는 정리입니다. 무한 곱의 수렴성은 이를 체크할 때 무한 곱이 어떤 값으로 수렴함을 보임과 동시에 적당히 큰 $N$에 대해서 $N$보다 큰 수열의 곱이 $0$이 아닌 값으로 수렴함을 보여야 하기 때문에 꽤나 까다로운 조건을 요구합니다. Weierstrass Factorization Theorem의 증명 과정에서는 이 무한 곱이 수렴함을 보임과 동시에 Compact set 위에서 Entire function으로 Uniformly converge하는 것을 보이기 위해 직접 $E_n$이 $\abs{z}\leq 1$에서 $\abs{1-E_n(z)}\leq \abs{z}^{n+1}$을 만족함을 보입니다. 이 정리는 Gamma function과 같은 특정 성질을 만족하는 복소 함수를 만들거나, Entire function의 집합이 B\'ezout domain을 만족함을 보일 때 등에서 사용되는 유용한 정리입니다. 제가 이 정리를 배울 때 인상깊었던 점은 이 정리가 Entire function들 사이에 multiplicity를 포함하여 영점이 같을 경우 서로 같은 부류의 함수로 생각할 수 있게 한다는 점이었습니다. 대수나 기하를 배우면서 한 point에서 함수가 영점이나 $\infty$를 가진다는 것은 그 점에서 singular할 수 있다는 것을 본 적이 있는데, 복소함수에서도 영점들이 함수의 성질에 영향을 미친다는 것이 흥미로웠기 때문입니다.

Weierstrass가 영점들을 통해 entire function을 분류한다면 Mittag-Leffler Theorem은 pole을 이용하여 meromorphic function을 만들 수 있게 하는 정리입니다. 좀 더 명확하게 하면 pole들이 열린 집합 $\Omega$에서 accumulate point를 가지지 않을 때 각각의 pole들에 대한 principal part-한 점 $a$를 pole로 가지는 $\frac{1}{z-a}$에 대한 다항식-으로 구성되고, 주어진 pole이 아닌 다른 pole이 없는 meromorphic function의 존재성을 밝히는 정리가 Mittag-Leffler Theorem입니다. 이 정리 또한 Weierstrass Factorization Theorem와 비슷하게 pole들을 이용하여 직접 meromorphic function을 만든다는 점에서 재미있었던 정리입니다.

위의 Weierstrass Factorization Theorem을 이용하여 $z=0, -1, -2, \ldots$에서 simple pole을 가지는 다음과 같은 함수를 만들 수 있습니다.
\begin{equation*}
\frac{1}{\Gamma(z)}=ze^{\gamma z}\prod\limits_{n=1}^\infty \left(1+\frac{z}{n}\right)e^{-\frac{z}{n}}
\end{equation*}
여기서 $\gamma$는 Euler constant입니다. 
위에서 정의한 Gamma function으로 다른 각도에서 보면 다음과 같이도 적을 수 있습니다.
\begin{equation*}
\Gamma(z)=\begin{cases}
\lim\limits_{n\rightarrow \infty}\frac{n^z n!}{z(z+1)\cdots (z+n)} & \text{for }z\notin \{0, -1, \ldots, \} \\
\int_0^\infty t^{z-1}e^{-t}~dt & \text{for Re}(z)>0
\end{cases}
\end{equation*}
더 나아가 Riemann-zeta function을 다음과 같이 정의한다면:
\begin{equation*}
\zeta(s)=\sum\limits_{s=1}^\infty \frac{1}{n^s}
\end{equation*}
다음과 같은 functional equation을 얻을 수 있습니다.
\begin{equation*}
\zeta(1-s)=2(2\pi)^{-s}\zeta(s)\Gamma(s)\cos\left(\frac{\pi}{2}s\right)
\end{equation*}
이번 수업 시간에 Gamma function과 Riemann-zeta function에 대한 많은 성질들을 복잡한 계산을 통해 구했고, 그 결과 $\zeta(s)$가 $0<\text{Re}(z)<1$에서는 어떻게 움직이는지 알 수 없지만, 그 외의 영역에선 $-2,-4, \ldots$에서만 zero를 가진다는 것을 보일 수 있었습니다.

이 Gamma function과 Riemann zeta function을 공부할 때 흥미로웠던 점은 해석학적 접근 방식이 불연속적인 정수론에 들어 맞는다는 점이었습니다. 해석학은 주로 metric space 위에서 논리를 전개하기 때문에 이산적인 정수론과 연결관계가 있을 것이라 생각하기 쉽지 않습니다. 또한 로지스틱 사상(Logistic map)과 같은 이산적인 계가 연속적인 계와 다른 카오스적 운동을 보이는 것처럼 실제로 이산적 계와 연속적 계의 움직임이 다른 경우도 존재합니다. 하지만 Riemann-zeta function은 소수와 큰 연관관계가 있음을 해석학적으로 접근한 것이 저에겐 큰 충격이었습니다.
\section{Riemann Mapping Theorem, Topological Remarks}
Riemann Mapping Theorem의 statement는 다음과 같습니다.(Complex Analysis, Lars V. Ahlfors 에서 가져왔습니다.)
\begin{theorem}
Given any simply connected region $\Omega$ which is not the whole plane, and a point $z_0\in \Omega$, there exists a unique analytic function $f(z)$ in$ \Omega$, normalized by the conditions $f(z_0)=0$, $f'(z_0)>0$, such that $f(z)$ defines a one-to-one mapping of $\Omega$ onto the disc $\abs{w}<1$.
\end{theorem}
이 정리는 증명 방식에서도 흥미롭고, 위상 수학적 관점에서도 인상적인 정리입니다. Ahlfors에 나오는 증명은 문제의 조건을 만족할 거 같은 좋은 함수들을 모은 다음에, 그 집합이 공집합이 아님을 보이고, 그 후 그 집합 중에서 가장 좋은 함수를 찾아서 위의 증명을 마무리하는 과정을 취하고 있습니다. 저는 이 증명을 공부하면서 이 증명방식은 정리가 맞다는 확신이 설 때가 아니면 이용하기 힘든 방식의 정리라고 생각하였습니다. 증명을 시작하기 위한 아이디어를 떠올리기도 어려울 뿐더러 constructive하지 않기 때문에 잘 보이지 않는 반례가 존재할 수 있기 때문입니다. 또한 열린 집합의 경계가 프랙탈과 같이 별로 좋지 못할 경우에도 이 정리가 성립해야하는데, 이 경우 많은 사람들이 정리가 성립함을 보이기 보다 반례를 찾는데 더 시간을 투자하지 않을까 싶습니다. 하지만 이 증명은 깔끔하고, constructive한 증명과 다른 느낌을 준다는 면에서 충분히 재미있고, 의미있다고 생각합니다.

이 정리가 위상 수학적 관점에서도 꽤 중요한 정리인 이유는, 전체 집합이 아닌 $\mathbb{C}$에서 simply connected open set이 unit disc와 biholomorphic함을 보이는 정리이며, 임의의 두 simply connected open set이 biholomorphic함도 보일 수 있습니다. Cauchy-Goursat Theorem와 더불어 Riemann Mapping Theorem은 복소 평면 위에서 Simply connected라는 집합의 조건이 중요하면서도 특별한 성질을 갖는다는 정보를 알려주는 정리들이라고 생각합니다.
\section{Harmonic functions and Dirichlet problem}
수업 마지막에 진행된 Harmonic function과 Dirichlet problem은 저에겐 복소함수로서의 느낌보다는 실변수 함수로써의 느낌이 좀 더 강한 부분이었습니다. $f$가 holomorphic이면 Real part가 조화함수이기 때문에 복소함수와 조화함수가 서로 연관된다는 사실은 알고 있었지만 $\rr^n$ 상에서 조화함수와 maximum principle, Harnark's inequality, green function을 학기 초반 편미분방정식 시간에 이미 배웠었기 때문입니다. 그래도 이 부분에서 흥미로웠던 점은 편미분방정식에서의 Green function을 이용하여 해를 구하는 방법이 아닌 Holomorphic function의 Real part를 이용하여 해를 구하는 방식으로도 복소 평면 위의 경계 조건 문제의 해-즉 Dirichlet problem-를 구할 수 있다는 점이었습니다.(이 때 Poisson kernel을 이용하였는데, 전자기학에서 사용하는 Methods of Image charge의 직관과 서로 연결되어 있어서 신기했습니다.) 또한 이 방법은 해의 영역이 달라지더라도 두 해의 영역 사이에 Biholomorphic function이 존재할 경우, 해를 자연스럽게 넘길 수 있다는 면에서 꽤 유용하다고 생각하였습니다. 예를 들어 저번 과제에서 disc위에서 경계 조건 문제의 해를 아는 상황에서, 이 해를 disc에서 Upper half plane으로 옮기는 Mobius Transformation을 이용하여 옮길 수 있음을 보였고, 이와 유사한 예시에서도 Conformal mapping을 통해 일반적으로 풀기 힘든 경계 조건에서도 해를 구할 수 있습니다.(실제로 전자기학에서 Conformal mapping을 이용하여 다양한 경계 조건 하에서 해를 구하는 문제가 있습니다.)

\section{마무리하며}
이번 학기동안 복소함수론과 실변수 함수론을 같이 배우면서 둘 다 같은 해석학적 논리를 사용함에도 함수를 바라보는 관점이 다름을 볼 수 있었습니다. 일변수 복소함수론에서는 실변수 함수론에서 주로 다루는 measurable function들보다 좋은 성질을 가지는 Holomorphic, meromorphic function들을 집중적으로 공부함에도 실변수 함수론 못지 않게 재밌는 현상들을 보여주고 더 나아가 편미분방정식, 위상수학, 대수기하, 정수론과도 연관되는 유용한 분야라는 것을 이번 학기를 통해 배울 수 있었습니다.

이번 학기 수고 많으셨습니다. 감사합니다.
\end{document}

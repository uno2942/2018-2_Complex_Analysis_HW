\documentclass{article}
\usepackage{graphicx, amssymb}
\usepackage{amsmath}
\usepackage{amsfonts}
\usepackage{amsthm}
\usepackage{kotex}
\usepackage{bm}
\usepackage{hyperref}
\usepackage{xcolor}
\usepackage{mathrsfs}
\usepackage{mathtools}
\usepackage{physics}
\usepackage{tikz}
\usetikzlibrary{decorations.markings}

\textwidth 6.5 truein 
\oddsidemargin 0 truein 
\evensidemargin -0.50 truein 
\topmargin -.5 truein 
\textheight 8.5in

\DeclareMathOperator{\cc}{\mathbb{C}}
\DeclareMathOperator{\rr}{\mathbb{R}}
\DeclareMathOperator{\bA}{\mathbb{A}}
\DeclareMathOperator{\fra}{\mathfrak{a}}
\DeclareMathOperator{\frb}{\mathfrak{b}}
\DeclareMathOperator{\frm}{\mathfrak{m}}
\DeclareMathOperator{\frp}{\mathfrak{p}}
\DeclareMathOperator{\slin}{\mathfrak{sl}}
\DeclareMathOperator{\Lie}{\mathsf{Lie}}
\DeclareMathOperator{\Alg}{\mathsf{Alg}}
\DeclareMathOperator{\Spec}{\mathrm{Spec}}
\DeclareMathOperator{\End}{\mathrm{End}}
\DeclareMathOperator{\rad}{\mathrm{rad}}
\newcommand*\Laplace{\mathop{}\!\mathbin\bigtriangleup}
\newcommand{\id}{\mathrm{id}}
\newcommand{\Hom}{\mathrm{Hom}}
\newcommand{\Sch}{\mathbf{Sch}}
\newcommand{\Ring}{\mathbf{Ring}}
\newcommand{\T}{\mathcal{T}}
\newcommand{\B}{\mathcal{B}}
\newcommand{\Mod}[1]{\ (\mathrm{mod}\ #1)}
\newtheorem{lemma}{Lemma}
\newtheorem{theorem}{Theorem}
\newtheorem{proposition}{Proposition}

\begin{document}
\title{Complex Analysis - HW2}
\author{SungBin Park, 20150462} 

\maketitle
\begin{enumerate}
\item[1.] Define
\begin{equation*}
F(s)=\int_0^\infty t^{s-1}e^{-t} dt
\end{equation*}
defined for $\text{Re }s>0$. Prove that
\begin{equation*}
\frac{F(s+n+1)}{n!n^s}\rightarrow 1
\end{equation*}
for $0< s\leq 1$ as $n\rightarrow \infty$.
\begin{proof}
Since $0<s\leq 1$, $t^s\leq n^s$ and $t^{s-1}\geq n^{s-1}$ iff $0\leq t\leq n$. Therefore,
\begin{equation*}
n^{s-1}\int_0^n t^{n+1} e^{-t} dt+n^{s}\int_n^\infty t^n e^{-t}dt \leq F(s+n+1)\leq n^s\int_0^n t^n e^{-t} dt+n^{s-1}\int_n^\infty t^{n+1} e^{-t}dt .
\end{equation*}
By integration by parts,
\begin{equation*}
\begin{split}
n^{s-1}\int_0^n t^{n+1} e^{-t} dt+n^{s}\int_n^\infty t^n e^{-t}dt &= n^{s-1}(n+1)\int_0^n t^{n} e^{-t} dt-e^{-n} n^{n+s}+n^{s}\int_n^\infty t^n e^{-t}dt \\
n^s\int_0^n t^n e^{-t} dt+n^{s-1}\int_n^\infty t^{n+1} e^{-t}dt &= n^s\int_0^n t^n e^{-t} dt+n^{s-1}(n+1)\int_n^\infty t^{n} e^{-t}dt + n^{n+s}e^{-n}.
\end{split}
\end{equation*}
Therefore,
\begin{equation*}
n^{s-1}\int_0^n t^{n} e^{-t} dt+n^{s}\int_0^\infty t^n e^{-t}dt -n^{n+s}e^{-n}\leq F(s+n+1)\leq n^s\int_0^\infty t^n e^{-t} dt+n^{s-1}\int_n^\infty t^{n} e^{-t}dt+n^{n+s}e^{-n}.
\end{equation*}
Since $\int_0^n t^ne^{-t}\leq n!$ and $\int_n^\infty t^ne^{-t}dt\leq n!$, as $n\rightarrow \infty$,
\begin{equation*}
1-\frac{n^{n}e^{-n}}{n!}\leq \frac{F(s+n+1)}{n^sn!}\leq 1+\frac{n^{n}e^{-n}}{n!}.
\end{equation*}
Using Taylor series of $e^n$,
\begin{equation*}
\frac{e^n n!}{n^n}\geq \frac{n!}{n^n}\left(\sum\limits_{i=n}^{2n} \frac{1}{i!}i^i\right)\geq \left(1+\frac{1}{2}+\cdots+\frac{1}{n}\right)
\end{equation*}
for large enough $n$, and it means that $\frac{e^{-n}n^n}{n!}\rightarrow 0$ as $n\rightarrow \infty$. Therefore,
\begin{equation*}
\frac{F(s+n+1)}{n^sn!}\rightarrow 1
\end{equation*}
as $n\rightarrow \infty$. The above inequality is established by
\begin{equation*}
\frac{(n+k)!}{n!}\leq (\sqrt[k]{k}n)^k=kn^k
\end{equation*}
for large enough $n$ and $n\leq k\leq 2n$.
\end{proof}

\item[2.] Show that
\begin{equation*}
\abs{\Gamma\left(\frac{1}{2}+it\right)}=\sqrt{\frac{2\pi}{e^t+e^{-t}}}
\end{equation*}
\begin{proof}
Since
\begin{equation*}
\Gamma(s)\Gamma(1-s)=\frac{\pi}{\sin\pi s}
\end{equation*}
for all $s$,
\begin{equation*}
\Gamma\left(\frac{1}{2}+it\right)\Gamma\left(\frac{1}{2}-it\right)=\frac{\pi}{\sin\left(\pi(\frac{1}{2}+it)\right)}=\frac{2\pi}{e^{\pi t}+e^{-\pi t}}
\end{equation*}
for all $t\in \rr$. As the real part is positive,
\begin{equation*}
\overline{\Gamma\left(\frac{1}{2}+it\right)}=\overline{\int_0^\infty e^{-u}u^{-\frac{1}{2}+it}du}=\int_0^\infty e^{-u}u^{-\frac{1}{2}-it}du=\Gamma\left(\frac{1}{2}-it\right).
\end{equation*}
Therefore,
\begin{equation*}
\abs{\Gamma\left(\frac{1}{2}+it\right)}=\sqrt{\frac{2\pi}{e^t+e^{-t}}}
\end{equation*}
\end{proof}

\item[3.] Calculate $\int_0^\infty t^{s-1}\cos t dt $ and $\int_0^\infty t^{s-1}\sin t dt$ for $0<\text{Re } s<1$.
\begin{proof}
Take a contour as Fig. \ref{Fig:P3}, then the contour integral for $z^{s-1}e^{-z}$ is
\begin{equation*}
\int_\epsilon^R t^{s-1}e^{-t} dt+\int_{C_R}z^{s-1}e^{-z} dz+\int_R^\epsilon (it)^{s-1} e^{-it} i~dt+\int_{C_\epsilon}z^{s-1}e^{-z} dz=0
\end{equation*}
since there is no pole in the interior of the contour. As $\epsilon\rightarrow 0$,
\begin{equation*}
\abs{\int_{C_\epsilon}z^{s-1}e^{-z} dz}\leq \epsilon^{s-1}\epsilon\frac{\pi}{2}=\epsilon^s \frac{\pi}{2}\rightarrow 0
\end{equation*}
As $R\rightarrow \infty$,
\begin{equation*}
\abs{\int_{C_R}z^{s-1}e^{-z} dz}\leq R^s \int_0^{\frac{\pi}{2}} e^{-R\cos\theta}d\theta\leq R^s \int_0^{\frac{\pi}{2}} e^{R(\theta-\pi/2)}d\theta\leq R^{s-1}\rightarrow 0.
\end{equation*}
The inequality is established by $\cos\theta\geq\pi/2-\theta$ for $0\leq\theta\leq\pi/2$.
Therefore,
\begin{equation*}
\int_0^\infty t^{s-1}e^{-t}dt=i^s\int_0^\infty t^{s-1}e^{-it}dt 
\end{equation*}
and $i^{-s}\Gamma(s)=\int_0^\infty t^{s-1}e^{-it}dt$.

Hence,
\begin{equation*}
\begin{split}
\text{Re}\int_0^\infty t^{s-1}e^{-it}~dt&=\int_0^\infty t^{s-1}\cos t~dt=\text{Re } e^{-\pi i s/2}\Gamma(s)=\cos(\pi s/2) \Gamma(s) \\
\text{Im}\int_0^\infty t^{s-1}e^{-it}~dt&=-\int_0^\infty t^{s-1}\sin t~dt=-\text{Im } e^{-\pi i s/2}\Gamma(s)=\sin(\pi s/2)\Gamma(s)
\end{split}
\end{equation*}
\begin{figure}[h]
\centering
\begin{tikzpicture}[decoration={markings,
mark=at position 1cm with {\arrow[line width=1pt]{>}},
mark=at position 3.5cm with {\arrow[line width=1pt]{>}},
mark=at position 6.1cm with {\arrow[line width=1pt]{>}}
}
]
% The axes
\draw[help lines,->] (-3,0) -- (3,0) coordinate (xaxis);
\draw[help lines,->] (0,-1) -- (0,3) coordinate (yaxis);

% The path
\path[draw,line width=0.8pt,postaction=decorate] (0.4,0) -- (2,0) node[below] {$R$} arc (0:90:2) -- (0,0.4);

\path[draw,line width=0.8pt,postaction=decorate] (0.4,0) node[below] {$\epsilon$} arc (0:90:0.4);

% The labels
\node[below] at (xaxis) {$x$};
\node[left] at (yaxis) {$y$};
\node[below left] {$O$};
\node at (1.5,1.8) {$C_{R}$};
\node at (0.5,0.45) {$C_{\epsilon}$};
\end{tikzpicture}
\caption{The contour used in problem 3.}
\label{Fig:P3}
\end{figure}

\end{proof}
\end{enumerate}

\end{document}

\documentclass{article}
\usepackage{graphicx, amssymb}
\usepackage{amsmath}
\usepackage{amsfonts}
\usepackage{amsthm}
\usepackage{kotex}
\usepackage{bm}
\usepackage{hyperref}
\usepackage{xcolor}
\usepackage{mathrsfs}
\usepackage{mathtools}
\usepackage{physics}
\usepackage{tikz}
\usetikzlibrary{decorations.markings}

\textwidth 6.5 truein 
\oddsidemargin 0 truein 
\evensidemargin -0.50 truein 
\topmargin -.5 truein 
\textheight 8.5in

\DeclareMathOperator{\cc}{\mathbb{C}}
\DeclareMathOperator{\rr}{\mathbb{R}}
\DeclareMathOperator{\bA}{\mathbb{A}}
\DeclareMathOperator{\fra}{\mathfrak{a}}
\DeclareMathOperator{\frb}{\mathfrak{b}}
\DeclareMathOperator{\frm}{\mathfrak{m}}
\DeclareMathOperator{\frp}{\mathfrak{p}}
\DeclareMathOperator{\slin}{\mathfrak{sl}}
\DeclareMathOperator{\Lie}{\mathsf{Lie}}
\DeclareMathOperator{\Alg}{\mathsf{Alg}}
\DeclareMathOperator{\Spec}{\mathrm{Spec}}
\DeclareMathOperator{\End}{\mathrm{End}}
\DeclareMathOperator{\rad}{\mathrm{rad}}
\newcommand*\Laplace{\mathop{}\!\mathbin\bigtriangleup}
\newcommand{\id}{\mathrm{id}}
\newcommand{\Hom}{\mathrm{Hom}}
\newcommand{\Sch}{\mathbf{Sch}}
\newcommand{\Ring}{\mathbf{Ring}}
\newcommand{\T}{\mathcal{T}}
\newcommand{\B}{\mathcal{B}}
\newcommand{\Mod}[1]{\ (\mathrm{mod}\ #1)}
\newtheorem{lemma}{Lemma}
\newtheorem{theorem}{Theorem}
\newtheorem{proposition}{Proposition}

\begin{document}
\title{Complex Analysis - HW4}
\author{SungBin Park, 20150462} 

\maketitle
\begin{enumerate}
\item[1.] If $f$ is holomorphic on the disk $D(0, R)$, $\abs{f}\leq M$ on the disc, and $f(z_0)=w_0$, then $f$ satisfies
\begin{equation*}
\abs{\frac{M(f(z)-w_0)}{M^2-\bar{w}_0 f(z)}}\leq \abs{\frac{R(z-z_0)}{R^2-\bar{z}_0 z}}
\end{equation*}
\begin{proof}
In the class, we showed that
\begin{equation*}
\phi_a(z)=\frac{z-a}{1-\bar{a}z}
\end{equation*}
is automorphism between unit open disc.
Let's construct $g$ as
\begin{equation*}
g(z)=\phi_{w_0/M}\circ \left(\frac{1}{M}f\right)\circ Rz \circ \phi_{-z_0/R}
\end{equation*}
For $z\in D(0, 1)$, $\abs{\frac{1}{M}f}\leq 1$ and $\abs{g(z)}\leq 1$. Also, $g(0)=0$. Therefore, using Schwarz lemma, we can know that
\begin{equation*}
\begin{cases}
\abs{g(z)}\leq \abs{z} \\
\abs{g'(0)}\leq 1.
\end{cases}
\end{equation*}
Therefore,
\begin{equation*}
\begin{split}
&\abs{\phi_{w_0/M}\circ \left(\frac{1}{M}f\right)\circ Rz \circ \phi_{-z_0/R}}\leq R\abs{z} \\
&\Rightarrow \abs{\phi_{w_0/M}\circ \left(\frac{1}{M}f\right)}\leq R\abs{\phi_{z_0/R}(z/R)} \\
&\Rightarrow \abs{\frac{\frac{1}{M}f-\frac{w_0}{M}}{1-\frac{\bar{w}_0}{M}\frac{f}{M}}}\leq \abs{\frac{z/R-z_0/R}{1-\bar{z}_0z/R^2}}\\
&\Rightarrow \abs{\frac{M(f-w_0)}{M^2-\bar{w}_0f}}\leq \abs{\frac{R(z-z_0)}{R^2-\bar{z}_0z}}
\end{split}
\end{equation*} 
Therefore,
\begin{equation*}
\abs{\frac{M(f(z)-w_0)}{M^2-\bar{w}_0f(z)}}\leq \abs{\frac{R(z-z_0)}{R^2-\bar{z}_0 z}}
\end{equation*}
\end{proof}

\item[2.] Describe all the automorphisms on upper half plane.(I'll denote upper half plane $\mathbb{P}$ and unit disc $\mathbb{D}$.)
\begin{proof}
I'll show that the automorphism of $\mathbb{P}$ is of form:
\begin{equation*}
f(z)=\frac{az+b}{cz+d}~~a,b,c,d\in \rr\text{ and }ad-bc=1.
\end{equation*}
Let $f$ is in the automorphism of $\mathbb{P}$. I'll assume that
\begin{equation*}
\phi(z)=\frac{z-i}{z+i}
\end{equation*}
is transformation from upper half plane to unit disc for now.(1) Then, $\phi\circ f(\mathbb{P})=\phi(\mathbb{P})=\mathbb{D}$ as a set-theoretic sense, and $\phi \circ f \circ \phi^{-1}$ is an automorphism of $\mathbb{D}$. We know that all the automorphism of $\mathbb{D}$ is of form:
\begin{equation*}
e^{i\theta}\frac{z-\alpha}{1-\bar{\alpha}z}
\end{equation*}
for some $\theta\in \rr$ and $\alpha\in \mathbb{D}$. I'll modify this form by
\begin{equation*}
\frac{pz+q}{\bar{q}z+\bar{p}}~~\text{for }p,q\in \mathbb{C}\text{ and }\abs{p}^2-\abs{q}^2=1.
\end{equation*}
I'll also assume this for now.(2) Then,
\begin{equation*}
\begin{split}
f(z)&=\phi^{-1}\circ \left(\frac{pz+q}{\bar{q}z+\bar{p}}\right)\circ \phi \\
&=\left(\frac{-iz-i}{z-1}\right)\circ\left(\frac{p\frac{z-i}{z+i}+q}{\bar{q}\frac{z-i}{z+i}+\bar{p}}\right) \\
&=\left(\frac{-iz-i}{z-1}\right)\circ \left(\frac{(p+q)z+(-p+q)i}{(\bar{p}+\bar{q})z+(\bar{p}-\bar{q})i}\right) \\
&=\frac{(p+\bar{p}+q+\bar{q})iz/\sqrt{8}-(\bar{p}-\bar{q}-p+q)/\sqrt{8}}{-(p+q-\bar{p}-\bar{q})z/\sqrt{8}-(-p-\bar{p}+q+\bar{q})i/\sqrt{8}}.
\end{split}
\end{equation*}
We know that $(z+\bar{z})i, (z-\bar{z})\in \rr$, so each coefficient in numerator and denominator are real. Let's rewrite it as
\begin{equation*}
\frac{az+b}{cz+d},~~a,b,c,d\in \rr
\end{equation*}
Also,
\begin{equation*}
\begin{split}
8(ad-bc)&=(p+\bar{p}+q+\bar{q})(p+\bar{p}-q-\bar{q})-(\bar{p}-p+\bar{q}-q)(p-\bar{p}-q+\bar{q}) \\
&=(p+\bar{p})^2-(\bar{p}-p)^2-(q+\bar{q})^2+(-q+\bar{q})^2=8(\abs{p}^2-\abs{q}^2)=8
\end{split}
\end{equation*}
Conversely, let if $f$ is such form, then
\begin{equation*}
\text{Im}(f)=\text{Im}\left(\frac{(az+b)(c\bar{z}+d)}{\abs{cz+d}^2}\right)=\frac{(ad-bc)y}{\abs{cz+d}^2}>0
\end{equation*}
for $z=x+iy$ and $y>0$. Also, $f$ is linear fractional transformation, so it is conformal mapping. Since it has inverse function on $\mathbb{P}$:
\begin{equation*}
f^{-1}(z)=\frac{-dz+b}{cz-a}
\end{equation*}
which is also satisfies $-d,b,c,-a\in \rr$, $ad-bc=1$, the domain and codomain is $\mathbb{P}$ and bijective. Therefore, $f$ is in the automorphism on $\mathbb{P}$, completing the proof.

(1): By geometric analysis, for any $z\in\mathbb{P}$,
\begin{equation*}
\frac{\abs{z-i}}{\abs{z+i}}<1
\end{equation*}
since the distance from $i$ is always smaller than the distance from $-i$. Also, the real axis is mapped to the boundary of unit disc since $\abs{z-i}=\abs{z+i}$. Since any linear fractional is one to one mapping of the extended $z$ plane to extended $w$ plane,(Complex Variables and Applications, James Ward Brown and Ruel V. Churchill) $\frac{z-i}{z+i}$ maps $\mathbb{P}$ to $\mathbb{D}$.

(2): Let's start from 
\begin{equation*}
f(z)=e^{i\theta}\frac{z-\alpha}{1-\bar{\alpha}z}.
\end{equation*}
Let $k=1-\abs{\alpha}^2>0$ and $a=e^{i\theta/2}/\sqrt{k}$, then
\begin{equation*}
\begin{split}
f(z)&=e^{i\theta}\frac{z-\alpha}{1-\bar{\alpha}z} \\
&=\frac{(e^{i\theta/2}/\sqrt{k})z-(e^{i\theta/2}/\sqrt{k})\alpha}{-(e^{-i\theta/2}/\sqrt{k})\bar{\alpha}z+(e^{-i\theta/2}/\sqrt{k})} \\
&=\frac{az+b}{\bar{b}+\bar{a}}
\end{split}
\end{equation*}
for $a=(e^{i\theta/2}/\sqrt{k})z$, $b=-(e^{i\theta/2}/\sqrt{k})\alpha$. Also, $\abs{a}^2-\abs{b}^2=\frac{1}{k^2}-\frac{\abs{\alpha}^2}{k^2}=1$.
\end{proof}
\end{enumerate}
\end{document}

\documentclass{article}
\usepackage{graphicx, amssymb}
\usepackage{amsmath}
\usepackage{amsfonts}
\usepackage{amsthm}
\usepackage{kotex}
\usepackage{bm}
\usepackage{hyperref}
\usepackage{xcolor}
\usepackage{mathrsfs}
\usepackage{mathtools}
\usepackage{physics}
\usepackage{tikz}
\usetikzlibrary{decorations.markings}

\textwidth 6.5 truein 
\oddsidemargin 0 truein 
\evensidemargin -0.50 truein 
\topmargin -.5 truein 
\textheight 8.5in

\DeclareMathOperator{\cc}{\mathbb{C}}
\DeclareMathOperator{\rr}{\mathbb{R}}
\DeclareMathOperator{\bA}{\mathbb{A}}
\DeclareMathOperator{\fra}{\mathfrak{a}}
\DeclareMathOperator{\frb}{\mathfrak{b}}
\DeclareMathOperator{\frm}{\mathfrak{m}}
\DeclareMathOperator{\frp}{\mathfrak{p}}
\DeclareMathOperator{\slin}{\mathfrak{sl}}
\DeclareMathOperator{\Lie}{\mathsf{Lie}}
\DeclareMathOperator{\Alg}{\mathsf{Alg}}
\DeclareMathOperator{\Spec}{\mathrm{Spec}}
\DeclareMathOperator{\End}{\mathrm{End}}
\DeclareMathOperator{\rad}{\mathrm{rad}}
\newcommand*\Laplace{\mathop{}\!\mathbin\bigtriangleup}
\newcommand{\id}{\mathrm{id}}
\newcommand{\Hom}{\mathrm{Hom}}
\newcommand{\Sch}{\mathbf{Sch}}
\newcommand{\Ring}{\mathbf{Ring}}
\newcommand{\T}{\mathcal{T}}
\newcommand{\B}{\mathcal{B}}
\newcommand{\Mod}[1]{\ (\mathrm{mod}\ #1)}
\newtheorem{lemma}{Lemma}
\newtheorem{theorem}{Theorem}
\newtheorem{proposition}{Proposition}

\begin{document}
\title{Complex Analysis - HW5}
\author{SungBin Park, 20150462} 

\maketitle
\begin{enumerate}
\item[1.] Let $f$ be a holomorphic function on $R_1<\abs{z}<R_2$ and continuous on $R_1\leq \abs{z}\leq R_2$. Let $M(r)=\max\limits_{\abs{z}=r}\abs{f(z)}$. Then,
\begin{equation*}
M(r)\leq M(R_1)^{\alpha}M(R_2)^{1-\alpha},~~\alpha=\log(R_2/r)/\log(R_2/R_1)
\end{equation*}
\begin{proof}
Consider $g(z)=z^\lambda f(z)$, $\lambda\in \rr$ and assume that this is not a constant function.(Unless, the theorem is trivial since the second paragraph directly applies to $f$.) Then, it is holomorphic on $0<R_1<\abs{z}<R_2$ except a branch cut. By Maximum modulus principle, $\abs{z^\lambda f}=\abs{z}^\lambda\abs{f}$ can not have maximum value inside the region without branch cut since $g$ is not a constant function. Since the $\abs{g}$ is independent from the setting of branch cut and continuous on the closure of the region, which is compact, $\abs{g}$ have maximum on the boundary of the region: $\abs{z}=R_1$ or $R_2$. Let each maximum values of $\abs{g}$: $R_1^\lambda M(R_1)$, $R_2^\lambda M(R_2)$.

We can fix $\lambda$ to satisfy $R_1^\lambda M(R_1)=R_2^\lambda M(R_2)$. Then, the $\lambda$ is $\log(M(R_2)/M(R_1))/\log(R_1/R_2)$. By Maximum modulus principle, we know that $r^\lambda M(r)\leq R_1^\lambda M(R_1)=R_2^\lambda M(R_2)$ for all $R_1\leq r\leq R_2$. Rewriting the inequality,
\begin{equation*}
\begin{split}
M(r)&\leq \left(\frac{R_2}{r}\right)^\lambda M(R_2) \\
&= \left(\frac{R_2}{r}\right)^{\log(M(R_2)/M(R_1))/\log(R_1/R_2)} M(R_2) \\
&=\left(\frac{R_2}{r}\right)^{\log_{R_1/R_2}(M(R_2)/M(R_1))} M(R_2) \\
&=\left(\frac{M(R_2)}{M(R_1)}\right)^{\log_{R_1/R_2}(R_2/r)} M(R_2) \\
&=\left(\frac{M(R_2)}{M(R_1)}\right)^{-\log(R_2/r)/\log({R_2/R_1})} M(R_2) \\
&=M(R_1)^\alpha M(R_2)^{1-\alpha}
\end{split}
\end{equation*}
for $\alpha=\log(R_2/r)/\log({R_2/R_1})$.
\end{proof}
\end{enumerate}
\end{document}
